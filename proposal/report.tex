\documentclass[12pt]{article}

\usepackage{amsmath}    % need for subequations
\usepackage{graphicx}   % need for figures
\usepackage{verbatim}   % useful for program listings
\usepackage{subfig}
\usepackage{hyperref}   % use for hypertext links, including those to external
                        % documents and URLs
\usepackage{soul}       % strike latex
\usepackage{amsthm}
\usepackage{amssymb}
\usepackage{wasysym}
\usepackage{relsize}

%% http://en.wikibooks.org/wiki/LaTeX/Page_Layout#Margins
\usepackage[margin=1in]{geometry}

%% Suppress page numbering
\pagenumbering{gobble}

\renewcommand{\comment}[1]{}

\usepackage{calc,amsmath}
\newlength\dlf
\newcommand\alignedbox[2]{
  % #1 = before alignment
  % #2 = after alignment
  &
  \begingroup
  \settowidth\dlf{$\displaystyle #1$}
  \addtolength\dlf{\fboxsep+\fboxrule}
  \hspace{-\dlf}
  \boxed{#1 #2}
  \endgroup
}

\newcommand{\ind}[3]{#1_{#2}^{(#3)}}
\newcommand{\alln}[1]{#1_{1:N}}
\newcommand{\Prod}{\mathlarger{\prod}}
\newcommand{\Sum}{\mathlarger{\sum}}

%% Fit long URLs in BibTex
\usepackage{hyperref}
\urlstyle{same}

\usepackage[usenames,dvipsnames]{color}
\definecolor{purple}{RGB}{128,0,128}
\allowdisplaybreaks

\begin{document}

\title{COS 511: Project Proposal}
\author{
  Yi-Hsien (Stephen) Lin\\ yihsien@princeton.edu
  \and
  Akshay Mittal \\ akshay@princeton.edu
}
\date{}

\maketitle

\noindent 

We are interested in exploring the connection between machine learning and game theory. In particular, how machine learning can benefit game theory such as solving a game or reaching an equilibrium more efficiently. One problem of many game theoretic applications we discovered while surveying over some paper is that they mostly assume complete
 availability of information in the games, which unfortunately, is not always the case. The paper ~\cite{Blum:2006} suggested that machine learning might be a solution to this problem. Also, repeated application of a machine learning
 algorithm to the game might speed-up the achievement of the
 equilibrium~\cite{Freund:1996}.

One particular game theorectic application we are interested in improving with machine learning techniques is online auction. Several research has already been conducted in this domain. ~\cite{He:2013} mentioned that sponsored search
 hosts an online auction among the advertisers to bid for the advertisements
 that the latter wish to present to the customers. The achievement of the
 equilibrium state in this case is also an application of the machine learning
 algorithms to game theory. 

\noindent As a part of this project study, we plan to study these three papers -
\cite{Freund:1996},~\cite{Blum:2006},~\cite{He:2013} - in detail (more related
 papers, if necessary) and attempt to improve upon the algorithms proposed. If
 our attempt is successful we plan to apply that algorithm to a real game and
 test the results and verify the validity of the algorithm compared to the
 state-of-the-art models.


{\footnotesize
\bibliography{report}
\bibliographystyle{unsrt}
}
\end{document}
